%-------------------------------------------------------------------------------
%	SECTION TITLE
%-------------------------------------------------------------------------------
\cvsection{Проекты}


%-------------------------------------------------------------------------------
%	CONTENT
%-------------------------------------------------------------------------------
\begin{cventries}

%---------------------------------------------------------
  \cventry
    {Java | MySQL | JDBC | Spark | Junit | Maven} % Role
    {Threatment App} % Event
    {\href{https://github.com/MasoNord/threatment-app}{GitHub}} % Location
    {} % Date(s)
    {
      \begin{cvitems} % Description(s)
        \item {Наличие тестов}
        \item {Получить запись об пациенте/пациентах}
        \item {Добавить нового пациента}
        \item {Обновить запись пациента}
        \item {Удалить запись пациента}
      \end{cvitems}
    }

%---------------------------------------------------------
  \cventry
    {Java | Gradle | Junit | Logback | Docker | redis-cli} % Stack
    {RedDB} % Project Name
    {\href{https://github.com/MasoNord/redDB}{GitHub}} % Link
    {} % empty
    {
      \begin{cvitems} % Description(s)
        \item {Асинхронная модель роботы}
        \item {Наличие юнит тестов}
        \item {Возможность запуска через Docker}
        \item {Реализованы следующие команды: ECHO, PING, EXISTS, GET, SET, QUIT, DEL}
      \end{cvitems}
    }

%---------------------------------------------------------
  \cventry
    {Socket Programming | Java | JDBC | Hikaricp | password4j | MySQL} % Stack
    {Chat Together} % Project Name
    {\href{https://github.com/MasoNord/chat-together}{GitHub}} % Link
    {} % empty
    {
      \begin{cvitems} % Description(s)
        \item {Многопоточкая модель работы}
        \item {Наличие авторизации}
        \item {Реализован в терминале}
      \end{cvitems}
    }
    \cventry
      {Java | PostgreSQL | Spring Boot | Docker | Junit | Mockito | Log4J | Lombok | Jackson | bucket4j | Hibernate | Maven} % Stack
      {Get On Time} % Project Name
      {\href{https://github.com/MasoNord/Get-On-Time}{GitHub}} % Link
      {} % empty
      {
        \begin{cvitems} % Description(s)
          \item {Наличие юнит тестов}
          \item {Наличие авторизации}
          \item {Реализован ограничитель на количество запросов от одного IP}
          \item {Наличие Swagger API}
          \item {Возможность сборки через docker-compose}
        \end{cvitems}
      }
%---------------------------------------------------------
\end{cventries}
